\chapter{第二章}

\section{第一节}

\subsection{第一小节}

\subsubsection{第一小小节}

图~\ref{fig:demo}。

\begin{figure}[H]
    \centering
    \includegraphics[width=0.3\linewidth]{figures/hnu-logo.png}
    \caption{示例图表}
    \label{fig:demo}
\end{figure}

三线表~\ref{table:demo}。

\begin{table}[H]
    \centering
    \caption{示例表格}
    \label{table:demo}
    \begin{tabular}{ccc}
        \toprule
        A   & B     & C     \\
        \midrule
        foo & 你好  & 世界  \\
        bar & Hello & World \\
        \bottomrule
    \end{tabular}
\end{table}


\subsection{第二小节}

\subsection{第三小节}

\subsubsection{第一小小节}

行内公式,$p = q * \frac{q}{p}$,$\begin{bmatrix} a & b & c \end{bmatrix}$。

单行公式。

\begin{equation}
    e = \lim_{n\to \infty} \left(1 + \frac{1}{n}\right)^n
\end{equation}

多行公式~\ref{eq:foo}。

\begin{equation}
    \begin{aligned}
        1+ 1*2 - (2-1) & = 1+ 2 - 1 \\
                       & = 3-1      \\
                       & = 2
    \end{aligned}
    \label{eq:foo}
\end{equation}

多行公式(无序号)。

\begin{equation*}
    \begin{aligned}
        1+ 1*2 - (2-1) & = 1+ 2 - 1 \\
                       & = 3-1      \\
                       & = 2
    \end{aligned}
\end{equation*}

\subsubsection{第二小小节}

引用 ResNet~\cite{he2016deep},中文引用~\cite{libaiThesis,chenjinbiao1980jixian}。


