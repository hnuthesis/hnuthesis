\chapter{绪\quad 论}

\section{研究背景以及意义}

人类活动持续排放的温室气体如CO2等导致了温室效应和全球变暖。尽管有越来越多的缓解气候变化的政策和措施,人类CO2排放量平均增长率在2000-2014年期间每年还是达到了2.6%,而在1970-2000年期间的年平均增长率却只有1.72%[1]。

\section{CO_2捕获技术}

\subsection{概述}

目前,CCS过程被认为是最具前途的温室气体减排方法,该过程旨在将大型工业设施所排放的CO2在进入大气之前进行捕获[8 , 9]。这些技术涉及从燃料燃烧或工业过程中捕获CO2,通过船舶或管道输送CO2,将其永久储存在地质构造深处,或进行石油采集驱替提高石油采收率。

\subsubsection{第二小小节}

